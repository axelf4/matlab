\documentclass{article}
\usepackage[T1]{fontenc}
\usepackage[utf8]{inputenc}
\usepackage{csquotes}
\usepackage[english]{babel}
\usepackage{amsmath, amsfonts, mathtools}
\usepackage{booktabs}
\usepackage{siunitx}
\usepackage{listings}
\usepackage{pgfplots, pgfplotstable}
\usepackage{lmodern, microtype}
\usepackage{enumitem}

\pgfplotsset{
	compat=1.15,
}
\usepgfplotslibrary{colorbrewer}
\lstset{
	breaklines=true,
	postbreak=\mbox{$\hookrightarrow$\space},
	breakindent = 5pt,
	breakautoindent = false,
}
\newtheorem{prop}{Proposition}
\renewcommand{\arraystretch}{1.2}
\sisetup{
	round-precision = 7,
	round-mode = figures,
}

\title{Windmills och shit}
\author{Axel Forsman, Jonas Lauri}

\begin{document}
\maketitle
Collectively approximately 27 hours were spent on this assignment.

We have seven locations that we want to consider for
the construction of offshore wind turbines.
To complicate matters, we will take the wake effects into consideration;
upstream wind turbines cause downstream wind speeds to decrease
as illustrated in figure~\ref{fig:wake_effects}.
With the data given in the assignment specification we will put together
a binary linear programming model for maximizing the power production,
and in turn, production revenue,
but, later, while also keeping the investment costs down.

\begin{figure}
	\centering
	\def\svgwidth{0.7\textwidth}
	\input{turbines.pdf_tex}
	\caption{Wake effects caused by an upstream wind turbine for the depicted wind direction. \label{fig:wake_effects}}
\end{figure}

\section{Production revenue}\label{sec:prod_revenue}
We assign indices $\mathcal L = \{1, \ldots, 7\}$ to the tentative locations,
see figure~\ref{fig:loc_numbering}, with the two distinct groups being
$ \mathcal G_1 = \{1, 2, 3\}, \mathcal G_2 = \{4, 5, 6, 7\}$.
Define $\mathcal E$ as the set of directed possible wind wake edges,
chosen so that for $(c, d) \in \mathcal E$ the angle between $c$ and $d$
is between west and east, exlusively and inclusively respectively.
The set of possible blade lengths is denoted
$$ \mathcal B = \{\text{long}, \text{medium}, \text{short}\} $$
Let $x_l^b$ be a logical variable whether
a turbine is installed at location $l \in \mathcal L$
with blade $b \in \mathcal B$,
and for all $(c, d) = e \in \mathcal E$ let $y_e^{ab}, \, a, b \in \mathcal B$
whether turbines with blades $a$ and $b$ are installed at $c$ and $d$ respectively.
Then the objective is
\begin{align*}
	\max_{x, y} \quad & f_\text{rev} \coloneqq p_\text{el}\sum_{\mathclap{d \in \text{Directions}}} \frac{\text{freq}(d)}{100} \left( \sum_{\substack{l \in \mathcal L \\ b \in \mathcal B}} \text{AE}(d, b) \, x_l^b - \sum_{\substack{e \in \mathcal E \\ a,b \in \mathcal B}} \text{WE}(d, e, a, b) y_e^{ab} \right) \\
	\text{s.t.} \quad & x, y \in \mathbb B \\
	& \sum_b x_l^b \le 1 \; \forall l \in \mathcal L, \quad \sum_{l, b} x_l^b \le n \\
	& x_c^a + x_d^b \le 1 + y_e^{ab}, \quad \forall (c, d) = e \in \mathcal E, \; a, b \in \mathcal B
\end{align*}
where $p_\text{el}$ is the price of electricity,
$$ \text{AE}(d, b) = \sum_{\mathclap{v \in \text{WindSpeeds}}} \mathbb P(v \vert \text{mws}(d)) \, \text{effect}(v, b), \quad b \in \mathcal B $$
is the average effect from wind in direction $d \in \text{Directions}$
from a turbine with blade $b$,
with the uncertainty in wind speed and direction integrated out, and
\begin{align*}
	\text{WE}(d, e, a, b) = &(1 - \text{relPowerLevel}(d, e, a)) \, \text{AE}(d, b) \\
	+ &(1 - \text{relPowerLevel}(d, -e, b)) \, \text{AE}(d, a), \\
	& d \in \text{Directions}, e \in \mathcal E, \; a,b \in \mathcal B
\end{align*}
is the average wake effects for direction $d$ associated with two turbines
across the edge $e$, equipped with blades $a$ and $b$ respectively.

\begin{figure}
	\centering
	\def\svgwidth{0.6\textwidth}
	\input{loc_numbering.pdf_tex}
	\caption{The numbering scheme used in defining the location node index set $\mathcal L$, and direction of the edges in $\mathcal E$. \label{fig:loc_numbering}}
\end{figure}

We consider three distinct policies for choosing blade sizes:
\begin{enumerate}[label=(\alph*)]
	\item Only medium blades: Add constraints
		$$ x_l^{[\text{long}, \text{short}]} = 0 \quad \forall l \in \mathcal L $$
	\item One type of blade for each group:
		Add variables $h_i^b, i = 1, 2; b \in \mathcal B$
		controlling what type of blade is allowed by turbines
		in each of the two groups,
		with constraints
		$$ x_l^b \le h_i^b \; \forall l \in \mathcal G_i, i = 1,2 \quad \sum_b h_i^b = 1 \forall i $$
	\item Any blade per turbine.
\end{enumerate}
The results are shown in table~\ref{tab:2_res}.
A turbine placed at location $l$ is represented
by the letter \texttt L, \texttt M, or \texttt S,
corresponding to the type of blade, or,
in the absence of a turbine, \texttt \_.
Looking at the objective values we can conclude that
the cases (a), (b), and (c) are in order of restrictiveness
since the costs are increasing.
We also see that large blades are used whenever possible. 
This as the increased production easily eclipses any losses due to increased wake effects.  
We see that for the case where no consideration was
given to wake effects there is no incentive to spread out turbines,
and in fact they are all adjacent.
The CPU times of all optimization runs were negligible.

\begin{table}
	\centering
	\caption{Results of the model in section~\ref{sec:prod_revenue},
	optimizing power production, for $n=3,4,5$ and the three cases:
	(a) Only medium blades;
	(b) One type of blade for each group;
	(c) Any blade per turbine,
	and with/without wake effects.
	The objective value is in \si{\kilo\watt}. \label{tab:2_res}}
	\renewcommand{\arraystretch}{1.2}
	\begin{tabular}{@{}l ccc c ccc@{}}
		\toprule
		& \multicolumn{3}{c}{Wake effects} && \multicolumn{3}{c}{No wake effects} \\
		\cmidrule{2-4} \cmidrule{6-8}
		& a & b & c && a & b & c \\ \midrule
		$n=3$ \\
		Obj & \num{743.879562} & \num{1572.346715} & \num{1572.346715} && \num{748.459854} & \num{1586.642336} & \num{1586.642336} \\
		Loc & \verb+_M_M_M_+ & \verb+L__L_L_+ & \verb+L__L_L_+ &&
		\verb+MMM____+ & \verb+___LLL_+ & \verb+LLL____+ \\

		$n=4$ \\
		Obj & \num{979.832117} & \num{2061.332117} & \num{2061.332117} && \num{997.945255} & \num{2115.521898} & \num{2115.521898} \\
		Loc & \verb+_MMM_M_+ & \verb+LL_L_L_+ & \verb+LL_L_L_+ &&
		\verb+MMMM___+ & \verb+___LLLL+ & \verb+LLLL___+ \\

		$n=5$ \\
		Obj & \num{1201.087591} & \num{2507.554745} & \num{2507.554745} && \num{1247.434307} & \num{2644.40146} & \num{2644.40146} \\
		Loc & \verb+_MMMMM_+ & \verb+LL_LLL_+ & \verb+LL_LLL_+ &&
		\verb+MMMMM__+ & \verb+LLLLL__+ & \verb+LLLLL__+ \\
		\bottomrule
	\end{tabular}
\end{table}

\section{Investment costs}
We add variables $g_1,g_2$ corresponding to whether
each of the two groups have been exploited,
and relate them to $x$ through the added constraints
$$ x_l^b \le g_i \quad \forall l \in \mathcal G_i, b \in \mathcal B, \quad i=1,2$$
where $\mathcal G_i$ contains the indices of the locations
in the respective group.
Then the investment costs become
$$ f_\text{inv} \coloneqq \sum_{l \in \mathcal L, b \in \mathcal B} (\text{locationCost} + \text{bladeCost}(b)) \, x_l^b + \text{groupCost} \, \sum_{i=1,2} g_i $$

If we try to minimize the investment costs with the added requirement
that the average production does not fall below
the best value obtained in 2(a) for $n=4$
(\num[round-mode=places,round-precision=2]{979.8336445553704})
we obtain as the optimal solution
$$ \text{Obj} = 80.0, \quad \text{Loc} = \verb+_LL____+ $$
As expected only one group is exploited.

\section{Multi-objective optimization}
We use the $\epsilon$\nobreakdash-constraint method,
and maximize the average revenue with
a variable cap $\epsilon$ on the maximum investment costs.
By minimizing and maximizing the investment costs in isolation
we can conclude that
$$ \epsilon \in [0, 271] $$
See figure~\ref{fig:pareto_front} which shows the objective plane
together with the Pareto front.
We find that it is approximately linear.
Since the production revenue is directly proportional to the
price of electricity any change to that quantity
would only alter the slope.
The placement of sampled solutions that are weak Pareto-optimal
follows a step function save for a few points,
as there is a minimum cost for the next production improvement,
i.e. new turbine or better blade.
The points in-between the larger steps are from additions of turbines
with blade other than the largest ones.
It is worth noting that a higher sample count would yield
more variety in production revenue -
the first solution with $f_\text{rev} \ne 0$ uses large blades for example.

\begin{figure}
	\centering
	\begin{tikzpicture}
		\begin{axis}[
				xlabel={$f_\text{inv}$, Investment costs}, ylabel = {$f_\text{rev}$, Production revenue},
				xmin=0, xmax=271,
				legend style={at={(0.02,0.98)}, anchor=north west},
				]
				\addplot[Set1-A, no marks] table[col sep=comma, x=InvestmentCost, y=ProductionRevenue] {5_pareto_front.csv};
				\addplot[Set1-B, only marks] table[col sep=comma, x=InvestmentCost, y=ProductionRevenue] {5_multi_obj.csv};
				\legend{Pareto frontier, Weak Pareto-optimal}
		\end{axis}
	\end{tikzpicture}
	\caption{Objective plane with Pareto front. \label{fig:pareto_front}}
\end{figure}

\section{Lagrangian relaxation}
For the model presented in section~\ref{sec:prod_revenue}
we Lagrangian relax the constraints on the $y$ variables,
$$ \underbrace{x_c^a + x_d^b - y_e^{ab}}_{\eqqcolon g_i(x, y)} \le \underbrace1_{\eqqcolon b_i} \forall e, a, b \implies b_i - g_i \ge 0 \, \forall i$$
that model the wake effects.
The Lagrangian subproblem becomes
\begin{align*}
	h(v) \coloneqq& \max_{x,y} f_\text{rev} + \sum v_i (b_i - g_i) \\
	=& \sum_i v_i + \max_x \sum_{l, b} (c_l^b - \sum_{\deg l} v_j) x_l^b + \sum \max (v_i - c_y^e) y_e
\end{align*}
where we have omitted repeating the other constraints,
interpolated $f_\text{rev}$ and simplified,
separated the problems of the $x$ and $y$ variables
and moved the maximum of the $y$ variables in under the sum,
since constraints are only on individual $y$:s.
If we start with initial value $v=0$ as is often done,
then for the first iteration all $y^{(0)}=0, x^{(0)}=1$,
which leads to the production of each turbine being unrealistically large
since no wake effects are taken into account.
In such situations, if we choose to update $v$ by means of gradient descent,
since the partial derivaties are
$$ h_i^{(k)} = \frac{\partial h}{\partial v_i} = 1 - \sum_{l \in \text{adj}(e), b} x_l^b + y_e $$
where the sum is over all locations adjacent to the edge $e$,
we will increase $v_i$ corresponding to the edges
where the turbines are installed on both sides yet $y_e=0$.

\appendix
\section{Julia code}\label{app:code}
\lstinputlisting{ass3.jl}

\end{document}
