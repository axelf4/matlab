\documentclass{article}
\usepackage[T1]{fontenc}
\usepackage[utf8]{inputenc}
\usepackage[swedish]{babel}
\usepackage[backend=biber]{biblatex}
\usepackage{hyperref}
\usepackage{siunitx}
\usepackage{booktabs}
\usepackage{crimson, microtype}

\bibliography{sources}
\sisetup{
  locale = DE,
  range-phrase = { till },
  per-mode = symbol,
}
\title{Inlämningsuppgift om hållbar utveckling\\ i kursen Miljö och matematisk modellering (MVE346), VT2021}
\author{Axel Forsman}

\begin{document}
\maketitle

De globala målen för hållbar utveckling är mål uppsatta av FN
för att mäta förbättring i levnadsstandard
som spänner tre huvudsakliga dimensioner: Den ekonomiska, ekologiska och sociala.
Här utvärderar vi de målen på tre sorters fordonsbränsle som används
för ändamålet persontransporter.

\section{Negativ påverkan för hållbar utveckling från utvalda drivmedel}

Vi identifier den negativa påverkan för hållbar utveckling
som uppkommer från användning för markburna persontransporter av
(I) bensin/diesel; (II) biodrivmedel; samt (III) el,
och möjliga åtgärder för att minska denna.
Den triviala tekniska åtgärden att öka motorns bränsleeffektivitet
går självklart att applicera på alla de tre drivmedlena,
och kommer därmed inte att diskuteras vidare. 

\subsection{Bensin/diesel}

Transportsektorn är nästan helt beroende av olja vilket kritiskt nog är
en icke-förnybar resurs \autocite{gudmundsson96}.
Därför leder det fortsatta bruket av bensin och diesel till att
de naturliga oljeresurserna minskar för kommande generationer.
Samtidigt som en oljekris är på horisonten, ökar transportefterfrågan
till västerländska standarder även i utvecklingsländer,
i takt med att deras ekonomiska ställning nödvändigtvis förbättras \autocite{edenhofer15}.
Därför riskerar en fördröjning med att investera i ny infrastruktur för
fordon med alternativa bränslen
att inte bara ekonomiska utan även sociala och kulturella mål
för hållbar utveckling inte kan uppnås.

Utsläppen från avgaser vid förbränning av bensin och diesel
blir en sorts baslinje vid jämförelser med biodrivmedel och eldrivna fordon.
Det är underförstått att dessa utsläpp har negativa ekologiska påverkningar
i att naturen används som en sänka med biodiversiteten som offer \autocite{gudmundsson96}.
Koldioxidutsläppen framförallt bidrar till global uppvärmning \autocite{edenhofer15, xu19} genom växthuseffekten.
Detta kan försökas åtgärdas med hjälp av \emph{Carbon Capture and Storage}-tekniker,
som fångar in CO2 antingen före eller efter förbränning.
Utöver det innehåller utsläppen hälsofarliga föroreningar.

\subsection{Biodrivmedel}

Vad gäller biodrivmedel är den bakomliggande tanken
att grödorna som används för framställningen har absorberat
just den mängd koldioxid som sedan frigörs vid förbränning.
På så sätt bör nettoutsläppet bli litet och
endast bestå av bidragen från produktionen och transporten av drivmedlet.
Enligt Energimyndigheten minskade all biodrivmedelsanvändning i Sverige år 2014
utsläppningen av koldioxidekvivalenter minst 46\% (för FAME)
jämfört med motsvarande fossila drivmedel \autocite{energimyndigheten15},
där andra typer av biobränslen var ännu bättre.
Dessa växthusgasutsläppsberäkningar tar i hänsyn hela biodrivmedelslivscykeln.

En studie från Internationella rådet för ren transport (ICCT) \autocite{omalley21}
fann att dagens biodieselblandningar försämrar luftkvaliteten jämfört med ren diesel,
med högre utsläpp av föroreningarna NO$_x$, HC och CO, se tabell~\ref{tab:biodiesel_emissions}.
Detta har negativ påverkan på dels sociala och ekologiska aspekter i försämrad hälsa.
Ökning av andelen biobränsle kan möjligen minska denna påverkan
då biobränsle har högre syreinnehåll vilket ökar förbränningseffektiviteten,
och då ökningen av NO$_x$ korrelerade med införandet av \emph{ultra-low sulfur diesel}.
Dessutom kan man optimera för lägre föroreningar i produktionen och motorns driftsförhållanden;
till exempel ger högre viskositet hos bränslet ökning av NO$_x$, HC och CO.

\begin{table}
  \centering
  \caption{Procentuell ökning av avgasutsläpp för biodieseln B20
    jämfört med petroleumdiesel under moderna förhållanden.
    Data från Internationella rådet för ren transport (ICCT) \autocite{omalley21}.
    \label{tab:biodiesel_emissions}}
  \begin{tabular*}{0.5\textwidth}{@{} l @{\extracolsep{\fill}} r @{}}
    \toprule
    \multicolumn{1}{c}{\textbf{Förorening}} & \multicolumn{1}{c}{\textbf{Ökning}} \\
    \midrule
    NO$_x$ & \SI{4}{\percent} \\
    PM & \multicolumn{1}{c}{obetydlig} \\
    HC & \SI{7}{\percent} \\
    CO & \SI{10}{\percent} \\
    \bottomrule
  \end{tabular*}
\end{table}

En studie om inverkan från biodiesel och bioethanol
på hållbar utveckling i den Europeiska unionen
fann att bioetanol haft en negativ påverkan på ekonomisk tillväxt
\autocite{simionescu17},
möjligen på grund av höga priser på jordbruksprodukter för några av grödorna
som ingår i bioetanoltillverkningen.
Detta specifika problem kan åtgärdas genom att
fokusera på tillverkning av biodiesel,
alternativt kanske på någon av de sorters biodrivmedel som inte beaktades.

Smil uppskattade att den genomsnittliga landkonsumptionen för produktionen
av biodrivmedel på \SI{0.30}{\watt\per\meter\squared}
är högre än för förnybara energikällor såsom vind, vatten och solkraft,
för vilka den är \SIlist[list-final-separator={ respektive }]{1; 3; 5}{\watt\per\meter\squared}
\autocite{smil15}.
Förbränning av solid biomassa från till exempel majsplantager
är mer effektivt än biobränslen där vätska extraherats,
då hela plantan går till användning.
Dessutom har tropiska plantager möjlighet att ge större skördar per hektar
än i andra klimat.
Den ökade uppbrukningen av landyta har negativ påverkan vad gäller hållbar utveckling
då habitatförstörelse leder till försämrad biodiversitet och ekologi
och mer ekonomiskt gynnsamma satsningar utesluts.

Vidare så leder skövling av skog till att CO2 tidigare bunden i vegetationen
eventuellt kommer hitta sin väg till atmosfären.
Jämförelse mellan CO2 utsläpp för olika biodieselblandningar,
rent biobränsle och diesel inkluderade,
måste därför beakta den areal som krävs för tillverkningen av biobränslet.

\subsection{Elektricitet}

Mer än de två kategorierna ovan är miljöpåverkningen från bränslet till elektriska fordon
direkt kopplad till sättet det har framställts på,
eftersom själva körningen av elektriska fordon sker utan avgaser \autocite{persson18}.
Därför hänger huruvida en övergång till elektriska fordon skulle leda till
lägre nivåer av utsläpp på om respektive lands energiproduktion är ``ren'' nog.
Till exempel uppfyller för närvarande inte Kinas eltillverkning inte det kravet
då den huvudsakligen består av att elda kol.
Alltså behövs det att fler länder övergår till renare el
för ett övergripande byte till elfordon med nettominskning av utsläpp.

För konsumenter är det dyrt att köpa ett elektriskt fordon.
En stor andel existerande fordon drivs med bensin, diesel eller biodrivmedel
och skulle i så fall behöva bytas ut.
Subventioner, skattebefrielser och andra incitament vilka är vanliga i europeiska länder
kan hjälpa till med att göra kostnaden för gemene man mindre.
Priset kan också antas sjunka i och med att massproduktion tar fart.

Utvinningen av litium som är huvudkomponenten i de batterier som ingår i dagens elektriska fordon
är inte utan problem.
När föroreningen vid gruvområdena tas i beaktning är det inte uppenbart huruvida
litiumbatterier minskar föroreningar globalt.
En sak är dock säker---om det skulle vara fallet skulle det ske på bekostning
av levnadsstandarderna för de bosatta runt gruvorna \autocite{persson18}.
Det är en väldigt vattenkrävande process som förorenar både luft och vatten
med giftiga kemikalier och avfall.
Konsekvenser kan uppstå för den lokala ekonomin på grund av minskad ekoturism.
Litium är även en icke-förnybar resurs.
Återvinning av litium sker i dagsläget i princip inte alls,
då priset på återvunnen litium är tre gånger högre än för nyutvunnen.
Istället återvinner en del företag bara de mer värdefulla metallerna kobalt och nickel
i batterierna medan litium endast ``downcyclas''.
Standardisering av batteriritningar kan göra återvinning mer lönsamt.
Vidare, till priset av dyrare elbilar, kan subventioner för återvinningsändamålet
göra att de ekologiska negativa påverkningarna minskas.

\section{Argumentation för framtida inriktning}

% Inledning
Transportsektorn står idag för \SI{27}{\percent} av den globala energiåtgången
och släpper ut \num{6.7} gigaton CO$_2$ per år \autocite{edenhofer15}.
Som det ser ut kommer CO$_2$ utsläppen att dubbleras till år 2050.
Med knappheten av oljan,
vilken är en nödvändig ingrediens i varje fordonsdrivmedel som idag är i bruk,
ser jag det som lämpligast att satsa på elektriska fordon för framtiden.

% Argument för
Olja är en begränsad resurs vars reserver håller på att ta slut redan nu.
Vid någon punkt i framtiden om det inte investeras i alternativa bränslen
kommer oljeprisökningen vara så stor att det varit lönsammare att byta helt
till elektriska fordon idag.
Att späda ut med biodrivmedel är bara att töja ut på tidshorisonten.
Då det ändå kommer behövs ske tids nog är det lika bra att börja lägga grunden idag.
Medan kostnaden för ett nätverk av uppladdningsstationer för elektriska fordon
är hög är det en bra investering för framtiden.
På detta sätt blir också till exempel de finansiella nackdelarna med
bland annat litiumbatterier mindre avgörande då
mer pengar naturligt kommer läggas åt batteritekniksforskning.

% Motargument
Möjligheten till att använda det ordinarie elnätet så som det ser ut idag
väger ej upp för föroreningarna associerade med litiumbatterier,
vilket kan vara en orsak till att inte investera i elektriska fordon.
Situationen är dock sådan att det även utan en elektrifierad transportsektor
kommer behöva göras satsningar för en förhöjd grad förnybar energi.
International Renewable Energy Agency uppskattar i sin senaste analys från 2019 \autocite{gielen19}
att \SI{27e12}[\$]{} kommer behövas investeras under perioden \numrange{2016}{2050}
för att ha en möjlighet att nå målen med Parisöverenskommelsen.
Parisöverenskommelsen, så klart, är ett beslut som försöker se till att
temperaturökningen från den globala uppvärmningen
inte orsakar oåterkalleliga skador på naturen,
vilket både skulle vara skadligt för naturens egenvärde
samt kunna ske inom så kort tidsspan att människan ej har tid
att undgå ett ogynnsamt öde.
Jag tar här en ekocentrisk ställning,
men förtydligar att vare sig hur man ser på substituerbarheten av ekonomiskt kapital i dagsläget,
så kommer naturkapital att bli den begränsande faktorn i framtiden om inte åtgärder vidtas.
Danmark är ett praktiskt exempel på att det är möjligt,
där \SI{60}{\percent} av den inhemska energitillförseln år 2018 genererades av förnybar energi
\autocite{energyindenmark2018}.
Alltså kan elektriska fordon ge en minskning i koldioxidutsläpp
betingad på en ökad andel förnybar elproduktion,
vilket är en icke-fråga då detta är en förutsättning för hållbar utveckling
oavsett val av fordonsbränsle.

Argument mot den omedelbara nöden av ett elbilsskifte är att vi kan
fortsätta använda förbränningsmotorer med fossila bränslen
så länge vardera hushåll ser till att ta sitt ansvar och
reducera deras körvanor.
Förenta nationernas klimatpanel rapporterar \autocite{edenhofer15} (med högt förtroende)
att allmänhetens acceptans av beslut ämnade att begränsa global uppvärmning
beror på den upplevda rättvisan gällande beslutsprocessen och konsekvenserna därav:
Att se [...] företag förorena utan straffa eller `freeloading'
kan bidra kraftigt till cynism och apati.
Alltså är konsumenterna villiga att göra uppoffringar för att nå hållbar utveckling,
och vad som saknas är företagens vilja att presentera mer hållbara alternativ.
Därav kan vi säga att det krävs politiska lösningar utöver individuellt ansvar
livsstilsförändringar inte kan väga upp.
Från ett historiskt perspektiv ser vi att företag, utan statlig reglering,
kommer fortsätta att förbruka finita resurser för att maximera vinst.
Det är då gemene man som sedan får ta smällen när han tvingas handla
oljebaserat drivmedel till skyhöga priser.
En proaktiv övergång till elektriska fordon förhindrar detta.

% Slutsats
Sammantaget utser dessa argument elektriska fordon som det bästa alternativet
för en övergripande transportsektorssatsning.
Detta kräver dock åtgärder på politisk nivå för att öka andelen förnybar energi.
Individen kan försöka se till att sin nästa bil blir en elbil,
och här kan till exempel politiska beslut för elbilssubventioner
och skatteökningar för förbränningsmotorer underlätta.

\printbibliography

\end{document}
