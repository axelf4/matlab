\documentclass{article}
\usepackage[T1]{fontenc}
\usepackage[utf8]{inputenc}
\usepackage[swedish]{babel}
\usepackage{amsmath}
\usepackage{comment}
\usepackage{tikz,tikz-3dplot}
\usepackage{titlesec}

\usetikzlibrary{calc}
\titleformat{\section}[runin]{\normalfont\itshape}{\thesection}{1em}{}[:]

\renewcommand{\vec}[1]{\mathbf{#1}}

\usetikzlibrary{3d}
\makeatletter
\tikzoption{canvas is xy plane at z}[]{%
	\def\tikz@plane@origin{\pgfpointxyz{0}{0}{#1}}%
	\def\tikz@plane@x{\pgfpointxyz{1}{0}{#1}}%
	\def\tikz@plane@y{\pgfpointxyz{0}{1}{#1}}%
	\tikz@canvas@is@plane}
\makeatother

\author{Axel Forsman}
\title{Projektion på plan}

\begin{document}
\maketitle

\section*{Problem}
Givet två icke-parallella vektorer $\vec{a}$ och $\vec{b}$ i det
tredimensionella rummet samt en godtycklig vektor $\vec{c}$,
bestäm projektionen $\vec{c'}$ av $\vec{c}$
på det plan som spänns av $\vec{a}$ och $\vec{b}$.
Förenkla uttrycket så lång som möjligt.
(Ledning: $\vec{c'}$ är den linjärkombination av $\vec{a}$ och $\vec{b}$
som har samma skalärprodukter med dessa som $\vec{c}$ har.)

\section*{Lösning}
Per ledning: Ansätt

$$ \vec{c'} = a \, \vec{a} + b \, \vec{b} $$


$
\hat{\vec{u}} = \frac{\vec{a} \times \vec{b}}
	{\lvert \vec{a} \times \vec{b} \rvert}
$
är en enhetsvektor ortogonal mot planet som spänns upp av $\vec{a}$ och $\vec{b}$.
Låt $\vec{c}_\perp = \hat{\vec{u}} \cdot \vec{c} \, \vec{c}$.
$\vec{c}_\perp$ är $\vec{c}$ projektion på $\vec{c}_\perp$ och därmed vinkelrät
mot planet.

\begin{figure}[htbp]
	\centering
	\tdplotsetmaincoords{70}{20}
	\begin{tikzpicture}[tdplot_main_coords]

		\tdplotsetrotatedcoords{340}{350}{25}
		\begin{scope}[tdplot_rotated_coords]
			\begin{scope}[canvas is xy plane at z=0]
				\fill[blue, fill opacity=0.1] (-2, -3) rectangle (2, 3);

				\draw[blue, ->] (0,0) -- (0.3,2) node[anchor=south west]{$\vec{a}$};
				\draw[blue, ->] (0,0) -- (1,0.2) node[anchor=west]{$\vec{b}$};

				\coordinate (cprime) at (0.5, 0.5);
			\end{scope}

			\coordinate (uhat) at (0, 0, 0.3);
			\coordinate (cperp) at (0, 0, 1.5);
			\coordinate (c) at ($(cprime) + (cperp)$);
		\end{scope}

		\draw[thick,->] (0,0,0) -- (1,0,0) node[anchor=north west]{$x$};
		\draw[thick,->] (0,0,0) -- (0,1,0) node[anchor=south]{$y$};
		\draw[thick,->] (0,0,0) -- (0,0,1) node[anchor=south]{$z$};

		% Draw help lines
		\draw[red, dotted, thin] (cperp) -- (c) -- (cprime);

		\draw[green, ->] (0,0) -- (cprime) node[anchor=south west]{$\vec{c'}$};
		\draw[green, ->] (0,0) -- (c) node[anchor=south west]{$\vec{c}$};
		\draw[green, ->] (0,0) -- (cperp) node[anchor=south east]{$\vec{c}_\perp$};

		\draw[->] (0,0,0) -- (uhat) node[anchor=south east]{$\hat{\vec{u}}$};
	\end{tikzpicture}
	\caption{Jävligt fin figur. \label{fig:fin}}
\end{figure}

\end{document}
