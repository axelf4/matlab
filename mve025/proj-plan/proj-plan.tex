\documentclass{article}
\usepackage[T1]{fontenc}
\usepackage[utf8]{inputenc}
\usepackage[swedish]{babel}
\usepackage{amsmath}
\usepackage{comment}
\usepackage{tikz, tikz-3dplot}
\usepackage{titlesec}
\usepackage{lmodern, microtype}

\usetikzlibrary{calc}
\titleformat{\section}[runin]{\normalfont\itshape}{\thesection}{1em}{}[:]

\renewcommand{\vec}[1]{\mathbf{#1}}

\usetikzlibrary{3d}
\makeatletter
\tikzoption{canvas is xy plane at z}[]{%
	\def\tikz@plane@origin{\pgfpointxyz{0}{0}{#1}}%
	\def\tikz@plane@x{\pgfpointxyz{1}{0}{#1}}%
	\def\tikz@plane@y{\pgfpointxyz{0}{1}{#1}}%
	\tikz@canvas@is@plane}
\makeatother

\author{Axel Forsman}
\title{Projektion på ett plan}

\begin{document}
\maketitle

\section*{Problem}
Givet två icke-parallella vektorer $\vec{a}$ och $\vec{b}$ i det
tredimensionella rummet samt en godtycklig vektor $\vec{c}$,
bestäm projektionen $\vec{c'}$ av $\vec{c}$
på det plan som spänns av $\vec{a}$ och $\vec{b}$.
Förenkla uttrycket så lång som möjligt.
(Ledning: $\vec{c'}$ är den linjärkombination av $\vec{a}$ och $\vec{b}$
som har samma skalärprodukter med dessa som $\vec{c}$ har.)

\section*{Lösning}
$\vec{\hat{u}} = \frac{\vec{a} \times \vec{b}}{\lvert \vec{a} \times \vec{b} \rvert}$
är en enhetsvektor ortogonal mot planet som spänns upp av $\vec{a}$ och $\vec{b}$.
Låt $\vec{c_\perp} = \text{projektionen av $\vec{c}$ på $\vec{\hat{u}}$}
	= \vec{\hat{u}} \cdot \vec{c} \, \vec{\hat{u}}$.
I och med ortogonal projektionens existens har vi $\vec{c} = \vec{c'} + \vec{c_\perp}$.
Lösning för $\vec{c'}$ ger
$$ \vec{c'} = \vec{c} - \left( \frac{\vec{a} \times \vec{b}}{\lvert \vec{a} \times \vec{b} \rvert^2}
	\, \cdot \vec{c} \right) \, \vec{a} \times \vec{b} $$

Per ledning ska $\vec{c'}$ ha samma skalärprodukter med $\vec{a}$ och $\vec{b}$ som $\vec{c}$,
vilket man lätt kan försäkra sig om,
och vara en linjärkombination av $\vec{a}$ och $\vec{b}$,
vilket följer av att $\vec{c'} \cdot \vec{\hat{u}} = 0$.

\begin{center}
	\tdplotsetmaincoords{70}{20}
	\begin{tikzpicture}[tdplot_main_coords]
		\tdplotsetrotatedcoords{340}{350}{25}
		\begin{scope}[tdplot_rotated_coords]
			\begin{scope}[canvas is xy plane at z=0]
				\fill[blue, fill opacity=0.1] (-2, -3) rectangle (2, 3);

				\draw[thick, blue, ->] (0,0) -- (0.3,2) node[anchor=south west]{$\vec{a}$};
				\draw[thick, blue, ->] (0,0) -- (1,0.2) node[anchor=west]{$\vec{b}$};

				\coordinate (cprime) at (0.5, 0.5);
			\end{scope}

			\coordinate (uhat) at (0, 0, 0.3);
			\coordinate (cperp) at (0, 0, 1.5);
			\coordinate (c) at ($(cprime) + (cperp)$);
		\end{scope}

		% Draw help lines
		\draw[red, dotted, thin] (cperp) -- (c) -- (cprime);

		\draw[red, ->] (0,0) -- (cprime) node[anchor=south west]{$\vec{c'}$};
		\draw[thick, ->] (0,0) -- (c) node[anchor=south west]{$\vec{c}$};
		\draw[->] (0,0) -- (cperp) node[anchor=south east]{$\vec{c_\perp}$};

		\draw[green, ->] (0,0,0) -- (uhat) node[anchor=south east]{$\vec{\hat{u}}$};
	\end{tikzpicture}
\end{center}
\end{document}
